\chapter{Kesimpulan dan Saran}

\section{Kesimpulan}
Berdasarkan penelitian yang telah dilakukan di dalam skripsi ini, disimpulkan bahwa:
\begin{enumerate}
	\item Masalah pencarian penempatan kamera CCTV berjumlah minimum yang mencakup seluruh ruangan dapat diselesaikan menggunakan metode linear programming. Masalah ini dapat memiliki banyak solusi, namun solusi-solusi tersebut belum tentu menghasilkan solusi kamera CCTV yang berjumlah minimum. Dengan memodelkan masalah ini dalam bentuk masalah linear programming, maka solusi optimal dari masalah ini dapat ditemukan.
	\item Perangkat lunak untuk menyelesaikan masalah ini telah berhasil dibangun. Perangkat lunak ini dapat menerima spesifikasi masalah dan menerapkan metode linear programming untuk menyelesaikannya. Hasil dari penyelesaian ini disajikan dalam bentuk visualisasi sehingga mudah untuk dipahami.
	\item Pengujian yang dilakukan menghasilkan informasi bahwa ukuran cell dan rasio antara sisi terpendek ruangan dengan jarak pandang kamera CCTV memiliki korelasi dengan tingkat \textit{overlap} dan \textit{out of bound}. Apabila ukuran cell semakin kecil, maka tingkat overlap dan out of bound akan semakin besar. Apabila rasio antara sisi terpendek ruangan dengan jarak pandang kamera CCTV semakin besar, maka tingkat \textit{overlap} dan \textit{out of bound} akan semakin kecil. Sedangkan untuk besar sudut pandang kamera CCTV, tidak ditemukan adanya korelasi dengan tingkat \textit{overlap} dan \textit{out of bound}.
\end{enumerate}
Dengan demikian, setiap tujuan dalam skripsi ini telah tercapai.

\section{Saran}
Berikut ini adalah saran-saran dari penulis bagi pembaca/peneliti yang ingin melanjutkan penelitian ini:
\begin{enumerate}
	\item Penerapan algoritma \textit{Heuristic} pada metode linear programming agar masalah linear programming dapat diselesaikan dengan lebih cepat.
\end{enumerate}