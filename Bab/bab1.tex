%versi 2 (8-10-2016) 
\chapter{Pendahuluan}
\label{chap:intro}
   
\section{Latar Belakang}
\label{sec:label}

Kamera merupakan alat/komponen optik yang digunakan untuk mengambil citra/gambar. Salah satu penggunaan kamera dalam kehidupan sehari-hari adalah kamera CCTV (\textit{closed-circuit television}). Kamera CCTV digunakan untuk memantau suatu lokasi dengan tujuan pengawasan dan keamanan. Kamera CCTV pada umumnya dipasang pada tempat strategis sehingga memiliki tingkat jangkauan yang baik. Kamera CCTV bekerja dengan cara merekam lokasi dalam jangkauannya secara terus menerus dan menyimpan hasil rekamannya dalam media penyimpanan. Rekaman ini biasanya digunakan oleh petugas keamanan untuk memantau lokasi tersebut dari tempat yang berbeda sehingga petugas tidak perlu memantau lokasi tersebut dengan datang secara langsung. Petugas hanya perlu mendatangi lokasi tersebut apabila mendapati hal-hal yang mencurigakan berdasarkan hasil rekaman CCTV.

Penempatan kamera CCTV di ruangan yang berbentuk sederhana (persegi panjang) tidaklah sulit. Kamera CCTV yang dibutuhkan biasanya berjumlah dua buah dan dipasang di kedua sudut ruangan yang merupakan satu diagonal. Namun, jika ruangan berukuran besar, maka tujuan penggunaan kamera CCTV pun tidak hanya untuk mendeteksi adanya orang, tetapi juga mengenali orang tersebut. Hal ini tentunya menyebabkan kesulitan dalam menentukan jumlah minimum dan lokasi penempatan kamera CCTV. Terdapat beberapa pendekatan yang dapat digunakan untuk menyelesaikan masalah ini, seperti dengan cara memasang kamera CCTV pada daerah-daerah yang dapat dimasuki orang. Tetapi pada kasus terburuk, orang bisa saja masuk melewati jalur-jalur yang tidak diduga, seperti tembok, atap, bawah tanah, dsb. Oleh karena itu, alangkah baiknya pemasangan kamera CCTV dilakukan hingga seluruh daerah di ruangan tersebut tercakup sepenuhnya.

Penempatan kamera CCTV dapat dilakukan dalam berbagai lokasi dan berbagai arah pandang. Apabila penempatan kamera CCTV dilakukan tanpa adanya perhitungan, maka terdapat kemungkinan jumlah kamera yang terlalu banyak dan/atau seluruh lokasi yang tidak tercakup sepenuhnya. Pada penempatan kamera CCTV terdapat perhitungan tingkat \textit{overlap} (penumpukan jangkauan) dan tingkat \textit{out of bound} (jarak pandang terpotong). Overlap merupakan bagian ruangam yang dicakup oleh lebih dari 1 kamera CCTV. \textit{Out of bound} merupakan cakupan kamera CCTV yang jatuh di luar ruangan.
%Penempatan kamera CCTV akan semakin baik apabila memiliki tingkat \textit{overlap} dan tingkat \textit{out of bound} yang semakin rendah.

Pada skripsi ini, akan dibuat sebuah perangkat lunak yang akan menghasilkan jumlah minimum beserta dengan penempatan-penempatan kamera CCTV berdasarkan ukuran ruangan, jarak pandang efektif kamera CCTV, dan sudut pandang kamera CCTV. Penempatan kamera CCTV akan terdiri dari lokasi penempatan dan juga sudut arah pandangnya. Perangkat lunak juga akan memberikan visualisasi yang berguna untuk membantu pengguna memahami penempatan setiap kamera CCTV. Hasil dari perangkat lunak ini dipastikan menjadi hasil yang paling optimal yang berarti bahwa kamera CCTV berjumlah paling minimum yang dapat mencakup seluruh 	ruangan.

\section{Rumusan Masalah}
\label{sec:rumusan}

Berdasarkan latar belakang masalah, maka ditetapkan rumusan masalah sebagai berikut:

\begin{itemize}
	\item Bagaimana cara menentukan jumlah minimum kamera CCTV dalam suatu ruangan?
	\item Bagaimana cara memvisualisasikan penempatan kamera-kamera CCTV dalam suatu ruangan?
\end{itemize}

\section{Tujuan}
\label{sec:tujuan}

Berdasarkan rumusan masalah, maka tujuan dalam skripsi ini adalah:

\begin{itemize}
	\item Mempelajari cara menentukan jumlah minimum kamera CCTV dalam suatu ruangan.
	\item Membangun perangkat lunak yang dapat mencari jumlah minimum kamera CCTV dan memvisualisasikan penempatan kamera-kamera CCTV dalam suatu ruangan.
\end{itemize}

\section{Batasan Masalah}
\label{sec:batasan}

Masalah yang dibahas dalam skripsi ini memiliki batasan-batasan sebagai berikut:

\begin{itemize}
	\item Ruangan dimodelkan dalam bidang 2 dimensi berbentuk persegi panjang.
	\item Spesifikasi kamera CCTV terdiri dari jarak pandang efektif dan besar sudut pandang.
\end{itemize}

\section{Metodologi}
\label{sec:metlit}
\begin{itemize}
	\item Mempelajari metode linear programming
	\item Mempelajari metode integer programming
	\item Mempelajari penggunaan kakas \textit{lp{\_}solve}
	\item Melakukan pemodelan masalah
	\item Menerapkan metode linear programming dalam penyelesaian masalah
	\item Merancang perangkat lunak
	\item Membangun perangkat lunak
	\item Menguji perangkat lunak
	\item Membuat kesimpulan
\end{itemize}

\section{Sistematika Pembahasan}
\label{sec:sispem}
\begin{itemize}
	\item \textbf{Bab 1 Pendahuluan}\\
	Pada bagian ini akan dijelaskan latar belakang masalah, rumusan masalah, tujuan dan batasan masalah. Terdapat penjelasan mengenai metodologi yang digunakan dalam melakukan penelitian ini.
	\item \textbf{Bab 2 Landasan Teori}\\
	Pada bagian ini terdapat pembahasan teori-teori yang digunakan untuk menyelesaikan masalah. Terdapat teori mengenai linear programming dan integer programming. Selain itu, terdapat penjelasan mengenai kakas \textit{lp{\_}solve} yang digunakan untuk membantu menyelesaikan masalah.
	\item \textbf{Bab 3 Analisis}\\
	Pada bagian ini terdapat penjelasan mengenai pemodelan masalah dan cara penyelesaian masalah menggunakan metode linear programming. Selain itu, terdapat analisis kebutuhan perangkat lunak yang terdiri dari diagram \textit{use case} dan diagram kelas sederhana.
	\item \textbf{Bab 4 Perancangan}\\
	Pada bagian ini dijelaskan bentuk perancangan antarmuka dan perancangan kelas diagram rinci yang digunakan untuk membangun perangkat lunak.
	\item \textbf{Bab 5 Pengujian}\\
	Pada bagian ini dibahas hasil-hasil pengujian yang dilakukan. Pengujian terdiri dari pengujian fungsional dan pengujian eksperimental.
	\item \textbf{Bab 6 Kesimpulan dan Saran}\\
	Pada bagian ini terdapat hal-hal apa saja yang didapatkan melalui penelitian ini. Selain itu, terdapat saran dari penulis bagi orang lain yang ingin melanjutkan penelitian ini.
\end{itemize}