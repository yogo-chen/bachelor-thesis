\chapter{Implementasi dan Pengujian}
Pada bab ini, akan dibahas hasil dari implementasi dan pengujian terhadap perangkat lunak yang dibangun. Pada saat tahap implementasi, terdapat spesifikasi lingkungan yang digunakan. Spesifikasi yang sama juga digunakan pada tahap pengujian. Pada tahap pengujian, terdapat 2 jenis pengujian, yaitu pengujian fungsional dan pengujian eksperimental.

\section{Lingkungan Implementasi Perangkat Keras}
Berikut ini merupakan spesifikasi perangkat keras yang digunakan pada tahap implementasi:
\begin{itemize}
	\item CPU: Intel\textsuperscript{\textregistered}{ }Core\texttrademark{ }i5-7200U Processor, 3M Cache, up to 3.10 Ghz
	\item GPU: NVIDIA GeForce 930MX
	\item RAM: 8GB
\end{itemize}

\section{Lingkungan Implementasi Perangkat Keras}
Berikut ini merupakan spesifikasi perangkat lunak yang digunakan pada tahap implementasi:
\begin{itemize}
	\item OS: Windows 10 Pro, 64-bit
	\item Pemrograman: Java 8 Update 152 (64-bit)
\end{itemize}

\section{Implementasi Antarmuka}
Pada bagian ini, akan dibahas hasil implementasi antarmuka sesuai dengan perancangan antarmuka yang dilakukan pada bab sebelumnya. Berikut ini adalah hasil dari implementasi antarmuka:
\begin{itemize}
	\item Antarmuka: \textbf{Penerima Masukan}\\
	Gambar~\ref{fig:ui_input} menunjukkan tampilan antarmuka penerima masukan. Pada antarmuka ini terdapat kolom-kolom masukan yang dapat diisi oleh pengguna. Apabila pengguna telah selesai mengisi kolom-kolom tersebut, pengguna dapat menekan tombol ''\textit{submit}'' untuk diarahkan ke antarmuka simulasi penempatan kamera CCTV.
	\begin{figure}[H]
		\centering  
		\includegraphics[scale=0.6]{ui_input}
		\caption[Antarmuka penerima masukan]{Antarmuka penerima masukan}
		\label{fig:ui_input}
	\end{figure}
	
	\item Antarmuka: \textbf{Simulasi Penempatan Kamera CCTV}\\
	Gambar~\ref{fig:ui_simulator} menunjukkan tampilan antarmuka simulasi penempatan kamera CCTV. Pada antarmuka ini, pengguna dapat melakukan simulasi penempatan kamera CCTV. Pengguna dapat melihat penempatan-penempatan beserta dengan cakupannya melalui panel visualisasi yang berada di bagian kanan antarmuka. Pada bagian kiri terdapat panel informasi yang menunjukkan informasi dari simulasi yang sedang dijalankan. Pada panel ini, pengguna dapat melihat penempatan-penempatan yang sedang diterapkan dalam ruangan. Pada bagian kanan atas antarmuka terdapat panel penambah kamera CCTV yang digunakan untuk menambah penempatan kamera CCTV. Pada panel ini, pengguna dapat menambah penempatan dengan menentukan koordinat dan sudut arah pandang dari kamera CCTV. Pengguna juga dapat memerintahkan simulasi untuk mencari penempatan-penempatan kamera CCTV dengan menggunakan metode yang dibahas pada bagian penyelesaian masalah. Hal ini dapat dilakukan pengguna dengan menekan tombol ''\textit{auto place camera}'' yang berada pada panel penambah kamera CCTV.
	\begin{figure}[H]
		\centering  
		\includegraphics[scale=0.45]{ui_simulator}
		\caption[Antarmuka penempatan kamera CCTV]{Antarmuka penempatan kamera CCTV}
		\label{fig:ui_simulator}
	\end{figure}
\end{itemize}

%\section{Pengujian Perangkat Lunak}
%Perangkat lunak yang telah diimplementasikan perlu diuji terlebih dahulu agar berjalan sebagaimana mestinya. Pada bagian ini akan dibahas 2 jenis pengujian yang dilakukan, yaitu pengujian fungsional dan pengujian metode penyelesaian masalah. Pada pengujian fungsional, perangkat lunak akan diuji agar bekerja sesuai fungsinya. Pada pengujian metode penyelesaian masalah, perangkat lunak akan diuji dengan eksperimen-eksperimen masalah untuk memastikan perangkat lunak dapat menyelesaikan masalah.

\section{Pengujian Fungsional}
Pada bagian ini, perangkat lunak akan diuji untuk memastikan bahwa setiap fungsi-fungsi dalam perangkat lunak dapat bekerja sesuai tujuannya. Pengujian ini dilakukan sesuai dengan skenario-skenario yang terdapat pada use case. Berikut ini pengujian-pengujian fungsional yang dilakukan:

\begin{itemize}
	\item Pengujian: \textbf{Memasukkan Spesifikasi Masalah}\\
	Pada pengujian ini, akan diuji apakah perangkat lunak dapat membangun simulasi masalah yang sesuai dengan masukan yang diberikan oleh pengguna. Pada pengujian ini, akan digunakan masukan sebagai berikut:	
	
	\begin{itemize}
		\item Lebar ruangan: 800 cm
		\item Panjang ruangan: 600 cm
		\item Jarak pandang kamera CCTV: 600 cm
		\item Besar sudut pandang kamera CCTV: \(90^\circ\)
		\item Ukuran terbesar cell: 100 cm
	\end{itemize}

	Masukan-masukan tersebut dimasukkan melalui antarmuka penerima masukan seperti pada gambar~\ref{fig:testing_input_before}.
	
	\begin{figure}[H]
		\centering  
		\includegraphics[scale=0.6]{ui_input}
		\caption[Tampilan pengisian masukan masalah]{Tampilan pengisian masukan masalah}
		\label{fig:testing_input_before}
	\end{figure}
	
	Setelah masukan-masukan tersebut dimasukkan, tombol ''\textit{submit}'' ditekan. Kemudian, perangkat lunak menampilkan antarmuka simulasi penempatan kamera CCTV seperti pada gambar~\ref{fig:testing_input_after}. 
	
	\begin{figure}[H]
		\centering  
		\includegraphics[scale=0.45]{ui_simulator}
		\caption[Tampilan setelah pengisian masukan masalah]{Tampilan setelah pengisian masukan masalah}
		\label{fig:testing_input_after}
	\end{figure}
	
	Pada panel informasi, terdapat informasi masalah yang sesuai dengan masukan yang diberikan. Panel informasi dapat dilihat pada gambar~\ref{fig:testing_input_side_pane_after}.
	
	\begin{figure}[H]
		\centering  
		\includegraphics[scale=0.6]{testing_input_side_pane_after}
		\caption[Tampilan panel informasi setelah pengisian masukan masalah]{Tampilan panel informasi setelah pengisian masukan masalah}
		\label{fig:testing_input_side_pane_after}
	\end{figure}	
	
	Hal ini menunjukkan bahwa proses memasukkan spesifikasi masalah telah berjalan sesuai dengan fungsinya. Dengan demikian, pengujian memasukkan spesifikasi masalah telah berhasil dilakukan.
	
	\item Pengujian: \textbf{Menambahkan Penempatan Kamera CCTV}\\
	Pada pengujian ini, akan diuji apakah perangkat lunak dapat merespon penambahan penempatan kamera CCTV dengan memperbaharui panel informasi dan panel visualisasi. Pengujian dilakukan dengan memilih titik (0, 0) dan mengisi sudut \(315^\circ\) sebagai posisi dan sudut arah pandang kamera CCTV. Gambar~\ref{fig:testing_add_placement_top_pane} menunjukkan tampilan pada saat mengisi penempatan kamera CCTV.
	
	\begin{figure}[H]
		\centering  
		\includegraphics[scale=0.6]{testing_add_placement_top_pane}
		\caption[Tampilan panel penambahan penempatan kamera CCTV]{Tampilan panel penambahan penempatan kamera CCTV}
		\label{fig:testing_add_placement_top_pane}
	\end{figure}
	
	 Setelah mengisi penempatan tersebut, tombol ''\textit{add camera}'' ditekan. Perangkat lunak merespon penambahan penempatan tersebut dengan memperbaharui panel informasi dan panel visualisasi seperti pada gambar~\ref{fig:testing_add_placement_side_pane_after} dan~\ref{fig:testing_add_placement_visualization_pane_after}. 
	 
	 \begin{figure}[H]
		\centering  
		\includegraphics[scale=0.6]{testing_add_placement_side_pane_after}
		\caption[Tampilan panel informasi setelah menambah penempatan kamera CCTV]{Tampilan panel informasi setelah menambah penempatan kamera CCTV}
		\label{fig:testing_add_placement_side_pane_after}
	\end{figure}
	
	\begin{figure}[H]
		\centering  
		\includegraphics[scale=0.45]{testing_add_placement_visualization_pane_after}
		\caption[Tampilan panel visualisasi setelah menambah penempatan kamera CCTV]{Tampilan panel visualisasi setelah menambah penempatan kamera CCTV}
		\label{fig:testing_add_placement_visualization_pane_after}
	\end{figure}
	 
	 Pada panel informasi bagian daftar penempatan kamera CCTV, terdapat baris baru bertuliskan ''\textit{At} (0, 0) \textit{facing} \(315^\circ\)'' yang menunjukkan penempatan pada titik (0, 0) dengan sudut arah pandang \(315^\circ\). Selain itu, informasi-informasi lainnya yang terdapat pada panel informasi juga telah diperbaharui. Pada panel visualisasi terdapat visualisasi objek kamera CCTV pada titik (0, 0) dengan sudut arah pandang \(315^\circ\). Hal ini menunjukkan bahwa penambahan penempatan kamera CCTV telah berhasil dilakukan sehingga proses menambahkan penempatan kamera CCTV dinyatakan berjalan sesuai dengan fungsinya. Dengan demikian, pengujian menambahkan penempatan kamera CCTV telah berhasil dilakukan.
	
	\item Pengujian: \textbf{Membuang Penempatan Kamera CCTV}\\
	Pada pengujian ini akan diuji apakah perangkat lunak dapat merespon pembuangan penempatan kamera CCTV dengan memperbaharui panel informasi dan panel visualisasi penempatan kamera CCTV. Pengujian dilakukan dengan memilih penempatan pada titik (0, 0) dengan sudut arah pandang (\(315^\circ\)) sebagai penempatan yang akan dibuang. Pada penempatan tersebut, tombol ''\textit{remove}'' ditekan. Perangkat lunak merespon pembuangan penempatan tersebut dengan memperbaharui panel informasi dan panel visualisasi seperti pada gambar~\ref{fig:testing_remove_placement_information_pane_after} dan~\ref{fig:testing_remove_placement_visualization_pane_after}. 
	
	\begin{figure}[H]
		\centering  
		\includegraphics[scale=0.6]{testing_input_side_pane_after}
		\caption[Tampilan panel informasi setelah membuang penempatan kamera CCTV]{Tampilan panel informasi setelah membuang penempatan kamera CCTV}
		\label{fig:testing_remove_placement_information_pane_after}
	\end{figure}
	
	\begin{figure}[H]
		\centering  
		\includegraphics[scale=0.5]{testing_remove_placement_visualization_pane_after}
		\caption[Tampilan panel visualisasi setelah membuang penempatan kamera CCTV]{Tampilan panel visualisasi setelah membuang penempatan kamera CCTV}
		\label{fig:testing_remove_placement_visualization_pane_after}
	\end{figure}	
	
	Pada panel informasi bagian daftar penempatan, tidak lagi ditemukan penempatan pada titik (0, 0) dengan sudut arah pandang \(315^\circ\). Pada panel visualisasi juga tidak lagi ditemukan visualisasi objek kamera CCTV pada titik (0, 0) dengan sudut arah pandang \(315^\circ\). Hal ini menunjukkan bahwa pembuangan penempatan kamera CCTV telah berhasil dilakukan sehingga proses membuang penempatan kamera CCTV dinyatakan berjalan sesuai dengan fungsinya. Dengan demikian, pengujian membuang penempatan kamera CCTV telah berhasil dilakukan.

\end{itemize}

%\section{Pengujian Perangkat Lunak}
%Pada bagian ini akan dibahas pengujian yang bertujuan kebenaran metode penyelesaian masalah yang telah dibahas sebelumnya. Pada pemodelan masalah, telah dibahas pemodelan ruangan ke dalam bentuk grid point sehingga daerah dalam ruangan dimodelkan dalam bentuk cell-cell. Selain itu juga, telah dibahas bahwa cakupan kamera CCTV dinyatakan dalam bentuk kumpulan cell sehingga mencakup seluruh ruangan berarti sama dengan mencakup seluruh cell pada ruangan. Pemodelan masalah ini digunakan dalam metode penyelesaian masalah menggunakan teknik linear programming yang tujuannya adalah mencari penempatan-penempatan kamera CCTV yang berjumlah paling minimum yang mencakup seluruh ruangan. Sehingga tujuan dari pengujian ini adalah menguji apakah tujuan dari metode penyelesaian masalah tersebut telah tercapai.

%Pada pengujian ini akan dilakukan eksperimen-eksperimen untuk memastikan bahwa perangkat lunak dapat menyelesaikan permasalahan sesuai dengan tujuannya. Terdapat beberapa pengujian perangkat lunak yang dilakukan, yaitu:

%Pengujian perangkat lunak akan dilakukan dengan melakukan eksperimen-eksperimen, sehingga hasilnya dapat dianalisa. Eksperimen akan dibagi menjadi 2 bagian utama. Pada bagian pertama akan dilakukan eksperimen untuk menganalisa tingkat \textit{overlap} dan \textit{out of bound}. Bagian kedua akan dilakukan eksperimen untuk menganalisa jumlah kamera minimum.
\section{Pengujian Eksperimental}
Pada bagian ini akan dibahas pengujian yang dilakukan dengan eksperimen. Eksperimen yang dilakukan bertujuan untuk mendapatkan informasi mengenai korelasi masukan spesifikasi masalah dengan tingkat \textit{overlap} dan \textit{out of bound}. Pada setiap eksperimen, akan dijalankan proses pencarian minimum kamera CCTV untuk mendapatkan hasil yang dapat dianalisa. Gambar tampilan perangkat lunak untuk setiap eksperimen dapat dilihat pada bagian lampiran.

\subsection{Eksperimen Ukuran Cell}
Pada eksperimen ini akan dianalisa tingkat \textit{overlap} dan \textit{out of bound} terhadap ukuran cell. Eksperimen ini dilakukan sebanyak 3 kali dengan ukuran cell yang berbeda. Berikut ini merupakan spesifikasi masalah yang digunakan dalam eksperimen ini:
\begin{enumerate}
	\item Eksperimen pertama
	\begin{itemize}
		\item Lebar ruangan: 800 cm
		\item Panjang ruangan: 600 cm
		\item Jarak pandang kamera CCTV: 275 cm
		\item Besar sudut pandang kamera CCTV: \(90^\circ\)
		\item Ukuran cell: 100 cm
	\end{itemize}
		
	\item Eksperimen kedua
	\begin{itemize}
		\item Lebar ruangan: 800 cm
		\item Panjang ruangan: 600 cm
		\item Jarak pandang kamera CCTV: 275 cm
		\item Besar sudut pandang kamera CCTV: \(90^\circ\)
		\item Ukuran cell: 50 cm
	\end{itemize}
		
	\item Eksperimen ketiga
	\begin{itemize}
		\item Lebar ruangan: 800 cm
		\item Panjang ruangan: 600 cm
		\item Jarak pandang kamera CCTV: 275 cm
		\item Besar sudut pandang kamera CCTV: \(90^\circ\)
		\item Ukuran cell: 25 cm
	\end{itemize}
\end{enumerate}
	
Berdasarkan eksperimen yang dilakukan, didapatkan hasil sebagai berikut:
\begin{enumerate}
	\item Eksperimen pertama
	\begin{itemize}
		\item Tingkat \textit{overlap} dan \textit{out of bound}: 0\%
	\end{itemize}
	
	\item Eksperimen kedua
	\begin{itemize}
		\item Tingkat \textit{overlap} dan \textit{out of bound}: 14.58\%
	\end{itemize}
	
	\item Eksperimen ketiga
	\begin{itemize}
		\item Tingkat \textit{overlap} dan \textit{out of bound}: 25\%
	\end{itemize}
\end{enumerate}

Berdasarkan hasil tersebut, didapatkan bahwa pemodelan menggunakan cell yang berukuran lebih kecil akan menghasilkan efektivitas yang lebih baik, yakni tingkat \textit{overlap} dan \textit{out of bound} yang lebih kecil.
%Berdasarkan hasil tersebut, didapatkan bahwa semakin besar ukuran terbesar cell, maka tingkat \textit{overlap} dan \textit{out of bound} akan semakin besar.
%Hal ini disebabkan karena semakin kecil ukuran cell, maka akurasi ketercakup cell semakin besar, sehingga tingkat \textit{overlap} dan \textit{out of bound} semakin besar.

\subsection{Eksperimen Rasio Sisi Terpendek Ruangan dengan Jarak Pandang Kamera CCTV}
Pada eksperimen ini akan dianalisa tingkat \textit{overlap} dan \textit{out of bound} terhadap rasio antara sisi terpendek ruangan dengan jarak pandang kamera CCTV. Eksperimen ini dilakukan sebanyak 4 kali dengan jarak pandang kamera CCTV yang berbeda sehingga didapatkan rasio yang berbeda. Berikut ini merupakan spesifikasi masalah yang digunakan dalam eksperimen ini:

\begin{enumerate}
	\item Eksperimen pertama
	\begin{itemize}
		\item Lebar ruangan: 400 cm
		\item Panjang ruangan: 300 cm
		\item Jarak pandang kamera CCTV: 300 cm
		\item Besar sudut pandang kamera CCTV: \(90^\circ\)
		\item Ukuran cell: 25 cm
		\item Rasio \(\frac{\text{sisi terpendek ruangan}}{\text{jarak pandang kamera CCTV}}\) = 1
	\end{itemize}
	
	\item Eksperimen kedua
	\begin{itemize}
		\item Lebar ruangan: 400 cm
		\item Panjang ruangan: 300 cm
		\item Jarak pandang kamera CCTV: 150 cm
		\item Besar sudut pandang kamera CCTV: \(90^\circ\)
		\item Ukuran cell: 25 cm
		\item Rasio \(\frac{\text{sisi terpendek ruangan}}{\text{jarak pandang kamera CCTV}}\) = 2
	\end{itemize}
	
	\item Eksperimen ketiga
	\begin{itemize}
		\item Lebar ruangan: 400 cm
		\item Panjang ruangan: 300 cm
		\item Jarak pandang kamera CCTV: 100 cm
		\item Besar sudut pandang kamera CCTV: \(90^\circ\)
		\item Ukuran cell: 25 cm
		\item Rasio \(\frac{\text{sisi terpendek ruangan}}{\text{jarak pandang kamera CCTV}}\) = 3
	\end{itemize}
	
	\item Eksperimen keempat
	\begin{itemize}
		\item Lebar ruangan: 400 cm
		\item Panjang ruangan: 300 cm
		\item Jarak pandang kamera CCTV: 75 cm
		\item Besar sudut pandang kamera CCTV: \(90^\circ\)
		\item Ukuran cell: 25 cm
		\item Rasio \(\frac{\text{sisi terpendek ruangan}}{\text{jarak pandang kamera CCTV}}\) = 4
	\end{itemize}
\end{enumerate}

Berdasarkan eksperimen yang dilakukan, didapatkan hasil sebagai berikut:
\begin{enumerate}
	\item Eksperimen pertama
	\begin{itemize}
		\item Tingkat \textit{overlap} dan \textit{out of bound}: 75\%
	\end{itemize}
		
	\item Eksperimen kedua
	\begin{itemize}
		\item Tingkat \textit{overlap} dan \textit{out of bound}: 16.67\%
	\end{itemize}
	
	\item Eksperimen ketiga
	\begin{itemize}
		\item Tingkat \textit{overlap} dan \textit{out of bound}: 8.33\%
	\end{itemize}
	\item Eksperimen keempat
	\begin{itemize}
		\item Tingkat \textit{overlap} dan \textit{out of bound}: 0\%
	\end{itemize}
\end{enumerate}
Berdasarkan hasil tersebut, didapatkan bahwa semakin besar rasio sisi terpendek ruangan dengan jarak pandang kamera CCTV, maka tingkat \textit{overlap} dan \textit{out of bound} akan semakin kecil.

\subsection{Eksperimen Besar Sudut Pandang Kamera CCTV}
Pada eksperimen ini akan dianalisa tingkat \textit{overlap} dan \textit{out of bound} terhadap besar sudut pandang kamera CCTV. Eksperimen ini dilakukan sebanyak 4 kali dengan besar sudut pandang kamera CCTV yang berbeda. Berikut ini merupakan spesifikasi masalah yang digunakan dalam eksperimen ini:

\begin{enumerate}
	\item Eksperimen pertama
	\begin{itemize}
		\item Lebar ruangan: 400 cm
		\item Panjang ruangan: 300 cm
		\item Jarak pandang kamera CCTV: 135 cm
		\item Besar sudut pandang kamera CCTV: \(90^\circ\)
		\item Ukuran cell: 25 cm
	\end{itemize}
	
	\item Eksperimen kedua
	\begin{itemize}
		\item Lebar ruangan: 400 cm
		\item Panjang ruangan: 300 cm
		\item Jarak pandang kamera CCTV: 135 cm
		\item Besar sudut pandang kamera CCTV: \(75^\circ\)
		\item Ukuran cell: 25 cm
	\end{itemize}
	
	\item Eksperimen ketiga
	\begin{itemize}
		\item Lebar ruangan: 400 cm
		\item Panjang ruangan: 300 cm
		\item Jarak pandang kamera CCTV: 135 cm
		\item Besar sudut pandang kamera CCTV: \(60^\circ\)
		\item Ukuran cell: 25 cm
	\end{itemize}
	
	\item Eksperimen keempat
	\begin{itemize}
		\item Lebar ruangan: 400 cm
		\item Panjang ruangan: 300 cm
		\item Jarak pandang kamera CCTV: 135 cm
		\item Besar sudut pandang kamera CCTV: \(45^\circ\)
		\item Ukuran cell: 25 cm
	\end{itemize}
\end{enumerate}

Berdasarkan eksperimen yang dilakukan, didapatkan hasil sebagai berikut:
\begin{enumerate}
	\item Eksperimen pertama
	\begin{itemize}
		\item Tingkat \textit{overlap} dan \textit{out of bound}: 14.58\%
	\end{itemize}
		
	\item Eksperimen kedua
	\begin{itemize}
		\item Tingkat \textit{overlap} dan \textit{out of bound}: 18.75\%
	\end{itemize}
	
	\item Eksperimen ketiga
	\begin{itemize}
		\item Tingkat \textit{overlap} dan \textit{out of bound}: 16.15\%
	\end{itemize}
	\item Eksperimen keempat
	\begin{itemize}
		\item Tingkat \textit{overlap} dan \textit{out of bound}: 8.33\%
	\end{itemize}
\end{enumerate}
Berdasarkan hasil tersebut, tidak ditemukan adanya korelasi besar sudut pandang kamera CCTV terhadap tingkat \textit{overlap} dan \textit{out of bound}.




















